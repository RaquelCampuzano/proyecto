% !TEX root =../LibroTipoETSI.tex

\begin{abstract}
Los ataques de Denegación de Servicio son, actualmente, uno de los mayores problemas a los que se
enfrentan los distintos equipos conectados a Internet. Todos ellos son víctimas potenciales de un
ataque DDoS: bien en el papel de equipos atacados o atacantes, bien de manera voluntaria o involuntaria. 
Estos ataques tienen un gran éxito e impacto económico por lo que son dirigidos a grandes empresas e instituciones y cada vez son más
baratos de llevar a cabo, por lo que parece que, a corto plazo, la situación con respecto a esta amenaza empeora.
En este artículo se presenta redborder DDoS como una medida de detección de este tipo de ataques haciendo uso del
algoritmo CUSUM como método de detección y de PF\_RING como forma de obtener un alto caudal de análisis
de paquetes.
\end{abstract}
