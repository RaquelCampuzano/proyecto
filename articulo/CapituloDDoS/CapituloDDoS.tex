% !TEX root =../articulo.tex
\section{Denegación de Servicio}\label{sec:Denegacion de Servicio}
\subsection{Definición}\label{ssec:dos definicion}

La definición formal que ofrece la \gls{ITU-T} en su recomendación X.800 es la siguiente \cite{ITU-T_DDoS_def}:

\begin{quote}
 The prevention of authorized access to resources or the delaying of time critical operations.
\end{quote}

Por su parte, el \gls{CNSS} ofrece una definición más general \cite{CCNS_DDoS_def}:

\begin{quote}
 Any action or series of actions that prevents any part of an information system from functioning.
\end{quote}

Así pues, una denegación de servicio impide directamente la disponibilidad de dicho servicio, ya sea debido o no 
a acciones maliciosas (ataque) originadas de manera local o remota y que afectan al sistema que se desea 
proteger. 
De esa forma, el que ofrece el recurso se queda sin los beneficios que el mismo genera durante un periodo de tiempo. %TODO: Reescribir.Esta última frase es muy confusa.  

\subsection{Taxonomía}\label{ssec:dos taxonomia}

La causa del ataque puede ser cualquier elemento de la comunicación: desde agotar
la capacidad del sistema servidor o el canal (ataque por inundación)
hasta un ataque dirigido a alguna vulnerabilidad del protocolo de red o la aplicación usada,
en cuyo caso hablamos de un ataque por una petición malformada.
\cite{Raghavan}. 
% TODO Destacar el tipo inundación en lugar del tipo 0-day?
Si bien el segundo tipo  de ataque requiere de muchos menos recursos computacionales para
ser llevado a cabo es necesario encontrar el fallo a explotar y el ataque dejará de estar
disponible en cuanto se corrija: con un sistema actualizado estamos menos expuestos
a este tipo de ataques.

Por su parte, se habla de un ataque \gls{DoS} distribuido si hay más de un elemento 
atacante\footnote{No confundir con los tipos \emph{Spoofing}, en los que se falsea el origen del ataque.} 
No es raro ver que en estos casos los atacantes son cientos o miles, y se han llegado a registrar ataques 
con millones de atacantes. Existen métodos para aumentar el éxito del ataque:
\begin{description}
  \item[Ataques sincronizados:] Muchos atacantes se sincronizan para acceder a un mismo recurso el
  mismo día.
  \item[Botnets:] Redes de nodos infectados que realizan el ataque controlado remotamente, sin conocimiento del
  dueño del nodo. Este tipo de redes se llegan incluso a alquilar o vender al atacante.
  \item[Sistemas de amplificación o reflexión:] EnUso de algún sistema para multiplicar el efecto del ataque,
  como el ataque \emph{Smurf}\index{Smurf}\footnote{Enviar un ping a una red entera con la dirección de respuesta 
  de la víctima.}
\end{description}

\subsection{Técnicas de detección y mitigación de un ataque DDoS}\label{sec:dos Deteccion y mitigación}
% TODO otro nombre para el apartado?
\subsubsection{Introducción}
Según el tipo de ataque (consultar la sección \ref{ssec:dos taxonomia}) la detección del mismo se realiza de un modo diferente.
Todos los casos adolecen de posibles falsos positivos y negativos. Además, son complementarios, es decir, (la mayoría de) los
ataques que un método detecta pasan inadvertidos por otro método.

\subsubsection{Ataques a vulnerabilidades}
Si el ataque está dirigido a un fallo en el canal de la comunicación o el servidor,
tendremos que examinar los paquetes de la comunicación a fin de encontrar los elementos de la
comunicación que buscan explotar ese fallo, es decir, la \emph{firma} del ataque\footnote{La firma no tiene
por qué ser sólo una expresión regular, sino que podría ser el enviar los paquetes de una forma determinada.}.
Este es el sistema de detección usado por IPS tales como snort \cite{snort} o suricata \cite{suricata}.

Como principales desventajas, es necesario mantener actualizada la base de datos de firmas, y esta
puede crecer en exceso haciendo que la búsqueda consuma recursos excesivos, siendo necesario entonces
seleccionar contra qué reglas vamos a analizar el flujo de datos.

\subsubsection{Ataques por inundación}\label{sssec:ddos_deteccion_inundacion}
Por su parte, si el ataque es por inundación, será necesario analizar el flujo en busca de anormalidades
en el perfil de tráfico, por ejemplo, mucho tráfico recibido. La principal dificultad de este método radica
en definir cuál es el tráfico ``normal'' de un sitio.

Un ejemplo claro es el efecto \emph{slashdot}\footnote{Efecto Barrapunto en español.}. El servidor de noticias
aguanta un tráfico enorme comparado con otros servidores pequeños. Cuando en el primero se crea una noticia que
enlaza al segundo, y pese a ser tráfico legítimo, el pequeño no puede aguantarlo y se produce la caída.

Es decir, no se puede definir un valor de tráfico ``normal'' válido para todos los servidores, sino que es
necesario hacer una medición del mismo a fin de detectar anomalías. Esto nos lleva a la segunda dificultad:
¿Qué parámetros debemos medir y cómo? Otra dificultad añadida comparada con el anterior método es 
seleccionar qué tráfico bloquear, ya
que si cortamos demasiado el ataque habrá tenido éxito gracias, precisamente, a la herramienta de defensa.

Es el método seguido por \redborderddos, y a lo largo del artículo expondremos las soluciones usadas para 
resolver dichas dificultades.

\endinput
