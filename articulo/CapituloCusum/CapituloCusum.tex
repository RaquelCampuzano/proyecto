% !TEX root =../articulo.tex
\section{El algoritmo CUSUM}\label{sec:CUSUM}

\subsection{Control de procesos}
El objetivo del \gls{CdP} es el de establecer un sistema de observación permanente e inteligente,
que detecte la aparición de variabilidad sobre la norma de un parámetro e identifique su origen, de forma que 
sea posible evitar su reaparición \cite{Control_de_procesos}.

\subsection{Algoritmo CUSUM}
El algoritmo CUSUM es un sistema de \gls{CdP} con memoria, es decir, la salida del
sistema no solo depende del estado actual de la variable de control, sino también del
estado anterior de la misma. De esta forma, restamos importancia a las desviaciones 
puntuales de la variable, y al mismo tiempo logramos resaltar los cambios pequeños
pero prolongados en el tiempo \cite{CUSUM_Carlos_III}.

Ante una Variable de Interés $x_i$, que sigue una distribución normal 
$x_i \sim\mathcal{N}\left(\mu_0,\sigma^2\right)$, 
construimos el estadístico $C_i$ que acumula la desviación de la variable sobre su media
a lo largo de las muestras:
\begin{align*}
 C_1 &= (x_1 - \mu_0) \nonumber\\
 C_2 &= (x_1 - \mu_0) + (x_1 - \mu_0) = C_1 + (x_2-\mu_0) \nonumber\\
 &\vdots& \nonumber\\
 C_i &= C_{i-1} + (x_i-\mu_0)
\end{align*}

\begin{figure}[htbp]
\includegraphics[width=\columnwidth]{CapituloCusum/Figuras/ejemploCusumControlado_crop}
\caption{Algoritmo CUSUM aplicado sobre un proceso controlado}
\label{fig:cusum_controlado} 
\end{figure}

\begin{figure}[htbp]
\includegraphics[width=\columnwidth]{CapituloCusum/Figuras/cusumDescontrolado-crop}
\caption{Algoritmo CUSUM aplicado sobre un proceso perturbado}
\label{fig:cusum_descontrolado} 
\end{figure}

Esto es, si las distintas muestras oscilan alrededor de $\mu_0$, $C_i$ tenderá a valer 0,
como vemos en \autoref{fig:cusum_controlado}. Por su parte, si las muestras presentan 
una desviación $k$ sobre $mu_0$, $C_i$ seguirá una recta $C_i=ik$ de pendiente $k$, 
como vemos en la \autoref{fig:cusum_descontrolado}. De esta forma, podemos definir una
variable descontrolada como aquella cuyo parámetro
$C_i$ ha superado un umbral definido para esa variable (normalmente, en forma de un
umbral de sensibilidad $C_i^+=M\sigma$).

Si queremos que las pequeñas variaciones no afecten al algoritmo, es posible también
definir un umbral mínimo sobre la desviación de la media $C_i^{min}=m\sigma$, de 
forma que sólo sumemos a $C_i$ las desviaciones superiores a $C_i^{min}$.

\subsection{Algoritmo CUSUM aplicado al control del tráfico}
\label{ssec:cusum_aplicado_trafico}
El tráfico de red es un proceso especialmente caótico, tanto por número de posibles
parámetros a controlar como la alta variabilidad de éstos,
haciendo imposible decidir un valor ``normal'' para éstos como ya se dijo en el 
\autoref{sssec:ddos_deteccion_inundacion}.

S.V. Raghvan y E.Dawson señalan los siguientes como los parámetros más importantes a la hora de detectar un ataque 
\gls{DoS}:

% TODO Quizás deberíamos decir: Todos en relación al número de paquetes IP excepto X, Y, Z?

\begin{enumerate}
 \item \gls{OWCD}, Densidad de conexiones de un solo sentido
 \begin{align*}
  \text{OWCD} = \frac{\sum\text{Paquetes OWC}}{\sum\text{Paquetes IP}}
 \end{align*}

 \item Longitud media de paquetes: 
   $\langle L\rangle_{\text{flow}} = \frac{\sum\text{Long paquetes IP}}{\sum\text{Paquetes IP}}$
 
 \item Relación paquetes entrantes y salientes: 
   $\frac{\sum\text{Paquetes entrantes}}{\sum\text{Paquetes salientes}}$
 
 \item \% de paquetes TCP: $\frac{\sum\text{Paquetes TCP}}{\sum\text{Paquetes IP}}$ 
 \item \% de paquetes UDP: $\frac{\sum\text{Paquetes UDP}}{\sum\text{Paquetes IP}}$
 \item Paquetes ICMP: $\frac{\sum\text{Paquetes ICMP}}{\sum\text{Paquetes IP}}$
 \item \% de paquetes LAND\footnote{Misma dirección origen y destino.}.
 \item Protocolo capa 4.
\end{enumerate}

\subsection{Algoritmo propuesto}
\begin{figure}[htbp]
\centering
\includegraphics[width=0.8\columnwidth]{CapituloCusum/Figuras/DiagramaEstados-crop}
\caption{Diagrama de estados del sensor}
\label{fig:diagrama_estados} 
\end{figure}

El sistema de defensa \redborderddos monitoriza periódicamente estos parámetros
para cada dirección IP externa a la red y para el tráfico normal que atraviesa
la misma. El sistema podrá estar en tres estados distintos, que se ven reflejados
en la \autoref{}

El primero de ellos es el de aprendizaje, al
inicio del programa y durante un tiempo determinado\footnote{Decidir dicho
periodo es importante para la precisión de la media obtenida, ya que incluso en
verano o invierno se puede observar diferencias en una red corporativa, por ejemplo.}
en el que el sistema averiguará los parámetros de los estadísticos descritos en la 
\autoref{ssec:cusum_aplicado_trafico}. Es importante que no ocurra ningún
ataque durante este periodo, ya que en ese caso el sistema tomará como normal
los valores de los estadísticos bajo ataque.

El segundo estado es el de defensa, en el que, durante un periodo determinado, el programa comparará
los parámetros en tiempo real contra la media obtenida en el anterior periodo y
unos parámetros de desviación $C_i^+$ y $C_i^{min}$ especificados en forma de múltiplos
de la desviación típica obtenida. Si durante este periodo no se detecta ningún ataque, los nuevos
valores $\mu_0$ y $\sigma$ obtenidos para cada estadístico pasarán a ser los estándares, de
forma que las variaciones normales no cuenten como un ataque. En otro
caso, el sistema pasará al modo alerta en el mismo momento que un estadístico supera su umbral.

Por último, tenemos el estado de alerta, al que se pasará cuando algún parámetro supere su umbral, y del que se
saldrá cuando los parámetros vuelvan a la normalidad. En este caso, se bloqueará el tráfico desde
las direcciones IP detectadas, y sólo cuando el parámetro vuelva a estar bajo el límite\footnote{Teniendo
en cuenta el tráfico bloqueado en el conteo de estadísticos.} se volverá al modo defensa.

\endinput
