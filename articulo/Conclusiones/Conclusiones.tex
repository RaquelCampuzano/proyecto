% !TEX root =../articulo.tex
\section{Conclusiones}

Tras haber investigado sobre la taxonomía de los distintos ataques \gls{DoS}, el método seguido para detectar parte de
ellos, el uso de PF\_RING como acelerador de captura y las diversas 
pruebas realizadas, plasmamos en esta sección las principales conclusiones.

\redborderddos{} Es un sistema de detección por descubrimiento de cambios en tráfico, esto es, se deben extraer 
las características \emph{normales} del tráfico, y analizar si el tráfico instantáneo entra dentro de esos parámetros o 
no. Por tanto, es incapaz de detectar ataques dirigidos a vulnerabilidades en los distintos elementos de la comunicación,
ya que no investiga el paquete a partir de la capa 4.

El algoritmo CUSUM es un sistema de control de procesos con memoria, esto es, un sistema de monitorización que permite 
detectar variaciones en la norma no solo basándose en el instante actual, sino en el comportamiento de la variable a lo 
largo del tiempo, sin requerir para ello almacenar todos sus estados anteriores.

De entre los posibles valores a monitorizar, se han escogido ocho valores que han demostrado ser representativos: 
Paquetes y bytes TCP, ICMP, UDP, densidad de conexiones de un solo sentido y relación entre paquetes entrantes y 
salientes.

Sin embargo, es necesario indicarle al algoritmo qué norma debe seguir el tráfico, y es la parte compleja de la 
situación. En las pruebas realizadas, el tráfico es sintético y predecible, pero el tráfico IP real varía mucho para un 
mismo segmento a diferentes horas del día, día de la semana, y periodos del año. Por ello, lo que se ha aprendido a una 
hora, no tiene por qué ser válido a la siguiente hora, y un tráfico legítimo puede ser clasificado rápidamente como 
ilegítimo. 

Así, CUSUM aplicado directamente no resulta una solución eficaz para la detección de anomalías en el tráfico de red y, 
por tanto, para detectar ataques \gls{DDoS}.

\endinput
