% !TEX root =../LibroTipoETSI.tex
%El anterior comando permite compilar este documento llamando al documento raíz
\section{Introducción}\label{introduccion}
El funcionamiento de la sociedad actual está cada vez más basado en 
información y, por tanto, en los sistemas que hacen posible su almacenamiento, procesado y comunicación. El bajo coste 
que supone hace que cada vez más empresas migren su infraestructura y operaciones a Internet.

Esta dependencia se hace palpable cuando estos servicios fallan, incluso cuando lo hacen por un breve periodo de tiempo. 
En él, el servicio ofrecido tiene un agujero en el que no produce, y, por tanto, el servicio no es capaz de generar 
dichos beneficios: El servicio no está disponible. Por otro lado, si el servicio está orientado a clientes externos, al 
observar el mismo que el servicio no está disponible, puede que piense que éste no es fiable o no ofrece un servicio 
de calidad, por lo que existe la posibilidad de generar desconfianza y rechazo.

Esta situación, en la que un cliente legítimo (esto es, con autorización para acceder al servicio) no es capaz de 
acceder a su servicio contratado debido a la acción intencionada de una persona, colectivo o empresa se denomina ataque 
de denegación de servicio (DoS attack). \cite{Raghavan}.

En los últimos años, internet ha experimentado un aumento significativo en la frecuencia e intensidad de ataques de 
Denegación de Servicio (DoS) \cite{kakaspersky_2H2011_DDoS_analisis}. Han afectado tanto a pequeños como a enormes 
sistemas, como Google \cite{Google+_DDoS_attack}, Amazon \cite{Amazon_DDoS_attack}, Paypal e incluso al FBI 
\cite{FBI_DDoS_attack}. El ataque más potente registrado hasta la fecha se produjo recientemente (entre el 18 y el 19 de 
marzo de 2013), y fue capaz de congestionar los routers de los proveedores TIER-1, esto es aquellos que
forman el nucleo de internet \cite{spamhaus_DDoS_attack}. 

A lo largo de este artículo describiremos cómo funciona \redborderddos, una herramienta diseñada 
para detectar y detener este tipo de ataques. Daremos la
definición de \gls{DoS} en la \autoref{sec:Denegacion de Servicio},
a fin de entender qué es y qué podemos esperar de \redborderddos{}
frente a ellos. A continuación, describiremos en la 
\autoref{sec:CUSUM} el algoritmo
CUSUM, usado por \redborderddos{} para detectar los ataques.

Con este conocimiento, describiremos cómo funciona 
\redborderddos{}, en qué tipo de red podemos usarlo, qué
tecnologías emplea para lograr un alto rendimiento y cómo está
estructurado en la \autoref{sec:estructura}.

Finalmente, describiremos las pruebas usadas, tanto en
laboratorio como en un entorno real, un ataque de \gls{DoS}
que ocurrió en Santander en mayo de 2013 en la
\autoref{sec:pruebas}.




% TODO Describir cada sección
%detallaremos y definiremos mejor lo que representa un 
%ataque de denegación de servicio y cómo se lleva a cabo en la sección \ref{sec:Tipos de ataques 
%DoS}, con el fin de poder justificar la estructura del programa descrita en la sección 
%\ref{sec:Estructura del programa y tecnologias usadas}. 

\endinput
