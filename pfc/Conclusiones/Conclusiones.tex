% !TEX root =../LibroTipoETSI.tex
\chapter{Conclusiones}\LABCHAP{Conclusiones}
\pagestyle{esitscCD}

\lettrine[lraise=-0.1, lines=2, loversize=0.25]{T}ras haber investigado sobre la taxonomía de los distintos ataques 
\gls{DoS}, el método seguido para detectar parte de ellos, el uso de PF\_RING como acelerador de captura y las diversas 
pruebas realizadas, plasmamos en este capítulo las principales conclusiones.

\section{Taxonomía de los ataques DDoS}
Un ataque \gls{DDoS} es toda acción que impida el acceso de usuarios legítimos a un servicio. de información. Podemos 
dividirlos, según su método de detección, en dos grandes grupos: Ataques por inundación, que son aquellos destinados a 
saturar los recursos de algún elemento de la comunicación, y ataques semánticos, que buscan explotar algún fallo en el 
algoritmo de los procesos implicados en la comunicación.

Para detectar los primeros, se deben usar sistemas de descubrimiento de cambios en tráfico, esto es, se deben extraer 
las características \emph{normales} del tráfico, y analizar si el tráfico instantáneo entra dentro de esos parámetros o 
no.

Por otro lado, la forma idónea de detectar ataques semánticos es mediante la búsqueda de patrones. Esto es, se debe
analizar el contenido de los paquetes, y detectar de esa forma paquetes con contenido malicioso.

\section{Algoritmo CUSUM}
El algoritmo CUSUM es un sistema de control de procesos con memoria, esto es, un sistema de monitorización que permite 
detectar variaciones en la norma no solo basándose en el instante actual, sino en el comportamiento de la variable a lo 
largo del tiempo, sin requerir para ello almacenar todos sus estados anteriores.

De entre los posibles valores a monitorizar, se han escogido ocho valores que han demostrado ser representativos: 
Paquetes y bytes TCP, ICMP, UDP, densidad de conexiones de un solo sentido y relación entre paquetes entrantes y 
salientes.

Sin embargo, es necesario indicarle al algoritmo qué norma debe seguir el tráfico, y es la parte compleja de la 
situación. En las pruebas realizadas, el tráfico es sintético y predecible, pero el tráfico IP real varía mucho para un 
mismo segmento a diferentes horas del día, día de la semana, y periodos del año. Por ello, lo que se ha aprendido a una 
hora, no tiene por qué ser válido a la siguiente hora, y un tráfico legítimo puede ser clasificado rápidamente como 
ilegítimo. 

Así, CUSUM aplicado directamente no resulta una solución eficaz para la detección de anomalías en el tráfico de red y, 
por tanto, para detectar ataques \gls{DDoS}.

\section{PF\_RING y ZeroCopy}
PF\_RING es una tecnología diseñada para sustituir el tratamiento normal que el núcleo de Linux da a los paquetes 
destinados a la monitorización pasiva, eliminando redundancias y deficiencias que el modelo actual tiene. 

PF\_RING funciona, principalmente, reservando de antemano muchos de los recursos que la máquina necesita para almacenar 
un paquete, y evitando que éste pase por la pila normal del núcleo, la cual tiene el inconveniente de necesitar, 
fundamentalmente, excesivas copias de memoria y demasiadas llamadas al sistema.

Por su parte, ZeroCopy es un marco de trabajo en el que los paquetes se pueden distribuir entre aplicaciones sin 
necesitar por ello realizar varias copias para cada una de ellas. Si funcionamos en modo ZeroCopy puro, además, 
evitamos la copia de memoria entre la tarjeta, el núcleo y el espacio de usuario, consiguiendo así muchos más recursos 
para el procesado de paquetes, misión principal del sensor.

En las pruebas realizadas, ambas tecnologías se muestran muy beneficiosas a la hora de analizar el tráfico de manera 
pasiva, pasando de no ser capaz de superar los 7000mbps en un enlace saturado usando el 100\% de la CPU a leer a 
velocidad de cable usando sólo un 4\% de la CPU.

\endinput

