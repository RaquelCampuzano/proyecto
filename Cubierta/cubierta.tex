% !TEX root =../LibroTipoETSI.tex
%El anterior comando permite compilar este documento llamando al documento raíz
\chapter{Instrucciones para la Cubierta y la Portada}\label{chp-01}
\epigraph{Aunque aquí se incluye la descripción de la cubierta y portada aprobadas por la ETSI, el presente formato está preparado para que introduzca los datos necesarios en el fichero principal, \ttcolor{pfcTipoETSI.tex}, y el compilador genere automáticamente la cubierta y portada siguiendo las directrices aprobadas.}%{Claude Shannon, 1948}

%\lettrine[lraise=0.7, lines=1, loversize=-0.25]{E}{n}
\lettrine[lraise=-0.1, lines=2, loversize=0.2]{L}{a} cubierta es la tapa del proyecto, mientras que la portada es la primera hoja que aparece al abrirlo. En Junta de Escuela de 25 de abril de 2014 se aprobó la obligatoriedad de utilizar la cubierta y portada que se incluyen en este ejemplo de formato y siguiendo las siguientes instrucciones. Debe modificar, en su caso y para la cubierta,
\begin{itemize}\itemsep1pt \parskip0pt \parsep0pt
\item la titulación, 
\item	el tipo de proyecto, atendiendo a si es fin de carrera, grado o máster.
\item el título del proyecto, 
\item el autor,
\item	el tutor o tutores,
\item el departamento,
\item y la fecha (año).
\end{itemize}

Para la portada, además de los anteriores, deberá cambiar el cargo del tutor. Por otro lado si el tutor no es docente de la ETSI entonces tendrá que añadir la figura de tutor ponente, que es un profesor de la ETSI encargado de realizar la gestión de la defensa.
El proyecto se escribirá en A4. En la cubierta, los diferentes campos se localizarán siguiendo el ejemplo de la cubierta en este documento. Tendrán los tamaños de letras y la posición orientativa siguientes, esta última dada en coordenadas en cm tomando como referencia la esquina inferior izquierda,
\begin{itemize}\itemsep1pt \parskip0pt \parsep0pt
\item la titulación 21 pt, (4.2,27)
el tipo de proyecto 21 pt, (4.2,25.9)
\item el título del proyecto 21 pt, (4.2,16.6), podrá subirse si el título excede dos líneas
\item el autor y tutor/es 15 pt, (4.2,13)
 \item el departamento, nombre de la ETSI y de la US, 14 pt y negrita, (centrado, 7.8). Si el nombre del departamento no cupiese en una línea, se utilizaría la siguiente, desplazando el texto inferior convenientemente.
\item para el texto “Sevilla, año” 13 pt, (centrado, 5.5)
\end{itemize}

Para la cubierta 
\begin{itemize}\itemsep1pt \parskip0pt \parsep0pt
\item la titulación 14 pt, (centrado,27.5)
\item el tipo de proyecto 14 pt, (centrado,26.9)
\item el título del proyecto 21 pt y negrita, (centrado,23)
\item el autor y tutor/es 11 pt, (centrado,19.5)
\item el departamento, nombre de la ETSI y de la US, 14 pt, (centrado 13)
\item para el texto “Sevilla, año” 11 pt, (centrado, 10.4)
\end{itemize}

La cubierta deberá incluir la imagen de fondo que se incluye en la cubierta de este texto, con las dos bandas vertical y horizontal en el color de la fachada del edificio Plaza América y la pequeña imagen de un detalle del edificio en la zona de cruce de las bandas. Incluirá el logotipo de la ETSI a la derecha del nombre de departamento. A pie de cubierta aparecerá el logo de la Universidad de Sevilla. El logo del departamento es opcional. Si no se incluyese, el de la Universidad de Sevilla se centraría en la hoja.

En la Sección \ref{sec:guía.cubierta}, encontrará alguna indicación de cómo puede modificar el aspecto de la portada. \textbf{En cualquier caso, el formato está preparado para que introduzca los datos necesarios en el fichero principal, \ttcolor{pfcTipoETSI.tex}, y el compilador genere automáticamente la cubierta y portada}.


\endinput
